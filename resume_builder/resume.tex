\documentclass{article}

\usepackage{hyperref}
\usepackage{graphicx}

\renewcommand{\baselinestretch}{1}
\setlength{\textheight}{9in}
\setlength{\textwidth}{6.5in}
\setlength{\headheight}{0in}
\setlength{\headsep}{0in}
\setlength{\topmargin}{0in}
\setlength{\oddsidemargin}{0in}
\setlength{\evensidemargin}{0in}
\setlength{\parindent}{.3in}

\begin{document}

\leftline{\tt Your First and Last Name}
\leftline{AME 20231}
\leftline{16 January 2025}

\medskip
This is a sample file in the text formatter \LaTeX.
I require you to use it for the following reasons:

\begin{itemize}

\item{It produces the best output of text, figures,
      and equations of any
      program I've seen.}

\item{It is machine-independent. It runs on Linux, Macintosh, and Windows machines. 
     You can e-mail {\tt ASCII} text versions of most relevant files.}

\item{It is the tool of choice for many research
     scientists and engineers.
     Many journals accept 
     \LaTeX~submissions, and many books
     are written in \LaTeX.}

\end{itemize}

\medskip
Some basic instructions are given next.
Put your text in here. You can be a little sloppy about
spacing. It adjusts the text to look good.

{\small You can make the text smaller.}  
{\tiny You can make the text tiny.}

Skip a line for a new paragraph.  
You can use italics ({\em e.g.} {\em Thermodynamics is everywhere}) or {\bf bold}.

Greek letters are easy: $\Psi$, $\psi$, $\Phi$, $\phi$.

A well known Maxwell thermodynamic relation is
\[
\left.{\partial T \over \partial P}\right|_{s} =
\left.{\partial v \over \partial s}\right|_{P}
\]

You can also set aside equations like so:
\begin{eqnarray}
du &=& T\ ds -P\ dv \qquad \mbox{(first law)} \label{fl}\\
ds &\ge& {\delta q \over T} \qquad \mbox{(second law)} \label{sl}
\end{eqnarray}

Eq.~(\ref{fl}) is the first law.  
Eq.~(\ref{sl}) is the second law.

References\footnote{
Lamport, L., 1986,
{\em \LaTeX: User's Guide \& Reference Manual},
Addison-Wesley: Reading, Massachusetts.
}
are available.

\medskip
\leftline{\em Running \LaTeX}
\medskip

{\em A modern option is the web-based} 
\href{https://www.overleaf.com}{\tt https://www.overleaf.com}.

Or you can create a \LaTeX~file with any text editor.
To get a document, you need to run the \LaTeX~application
on the text file.

\medskip
{\tt pdflatex file.tex}

\medskip
This generates auxiliary files and a {\tt file.pdf}, which can be viewed using standard PDF viewers.

\end{document}
